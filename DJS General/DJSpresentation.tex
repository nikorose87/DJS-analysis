
% Default to the notebook output style

    


% Inherit from the specified cell style.




    
\documentclass[11pt]{article}

    
    
    \usepackage[T1]{fontenc}
    % Nicer default font (+ math font) than Computer Modern for most use cases
    \usepackage{mathpazo}

    % Basic figure setup, for now with no caption control since it's done
    % automatically by Pandoc (which extracts ![](path) syntax from Markdown).
    \usepackage{graphicx}
    % We will generate all images so they have a width \maxwidth. This means
    % that they will get their normal width if they fit onto the page, but
    % are scaled down if they would overflow the margins.
    \makeatletter
    \def\maxwidth{\ifdim\Gin@nat@width>\linewidth\linewidth
    \else\Gin@nat@width\fi}
    \makeatother
    \let\Oldincludegraphics\includegraphics
    % Set max figure width to be 80% of text width, for now hardcoded.
    \renewcommand{\includegraphics}[1]{\Oldincludegraphics[width=.8\maxwidth]{#1}}
    % Ensure that by default, figures have no caption (until we provide a
    % proper Figure object with a Caption API and a way to capture that
    % in the conversion process - todo).
    \usepackage{caption}
    \DeclareCaptionLabelFormat{nolabel}{}
    \captionsetup{labelformat=nolabel}

    \usepackage{adjustbox} % Used to constrain images to a maximum size 
    \usepackage{xcolor} % Allow colors to be defined
    \usepackage{enumerate} % Needed for markdown enumerations to work
    \usepackage{geometry} % Used to adjust the document margins
    \usepackage{amsmath} % Equations
    \usepackage{amssymb} % Equations
    \usepackage{textcomp} % defines textquotesingle
    % Hack from http://tex.stackexchange.com/a/47451/13684:
    \AtBeginDocument{%
        \def\PYZsq{\textquotesingle}% Upright quotes in Pygmentized code
    }
    \usepackage{upquote} % Upright quotes for verbatim code
    \usepackage{eurosym} % defines \euro
    \usepackage[mathletters]{ucs} % Extended unicode (utf-8) support
    \usepackage[utf8x]{inputenc} % Allow utf-8 characters in the tex document
    \usepackage{fancyvrb} % verbatim replacement that allows latex
    \usepackage{grffile} % extends the file name processing of package graphics 
                         % to support a larger range 
    % The hyperref package gives us a pdf with properly built
    % internal navigation ('pdf bookmarks' for the table of contents,
    % internal cross-reference links, web links for URLs, etc.)
    \usepackage{hyperref}
    \usepackage{longtable} % longtable support required by pandoc >1.10
    \usepackage{booktabs}  % table support for pandoc > 1.12.2
    \usepackage[inline]{enumitem} % IRkernel/repr support (it uses the enumerate* environment)
    \usepackage[normalem]{ulem} % ulem is needed to support strikethroughs (\sout)
                                % normalem makes italics be italics, not underlines
    

    
    
    % Colors for the hyperref package
    \definecolor{urlcolor}{rgb}{0,.145,.698}
    \definecolor{linkcolor}{rgb}{.71,0.21,0.01}
    \definecolor{citecolor}{rgb}{.12,.54,.11}

    % ANSI colors
    \definecolor{ansi-black}{HTML}{3E424D}
    \definecolor{ansi-black-intense}{HTML}{282C36}
    \definecolor{ansi-red}{HTML}{E75C58}
    \definecolor{ansi-red-intense}{HTML}{B22B31}
    \definecolor{ansi-green}{HTML}{00A250}
    \definecolor{ansi-green-intense}{HTML}{007427}
    \definecolor{ansi-yellow}{HTML}{DDB62B}
    \definecolor{ansi-yellow-intense}{HTML}{B27D12}
    \definecolor{ansi-blue}{HTML}{208FFB}
    \definecolor{ansi-blue-intense}{HTML}{0065CA}
    \definecolor{ansi-magenta}{HTML}{D160C4}
    \definecolor{ansi-magenta-intense}{HTML}{A03196}
    \definecolor{ansi-cyan}{HTML}{60C6C8}
    \definecolor{ansi-cyan-intense}{HTML}{258F8F}
    \definecolor{ansi-white}{HTML}{C5C1B4}
    \definecolor{ansi-white-intense}{HTML}{A1A6B2}

    % commands and environments needed by pandoc snippets
    % extracted from the output of `pandoc -s`
    \providecommand{\tightlist}{%
      \setlength{\itemsep}{0pt}\setlength{\parskip}{0pt}}
    \DefineVerbatimEnvironment{Highlighting}{Verbatim}{commandchars=\\\{\}}
    % Add ',fontsize=\small' for more characters per line
    \newenvironment{Shaded}{}{}
    \newcommand{\KeywordTok}[1]{\textcolor[rgb]{0.00,0.44,0.13}{\textbf{{#1}}}}
    \newcommand{\DataTypeTok}[1]{\textcolor[rgb]{0.56,0.13,0.00}{{#1}}}
    \newcommand{\DecValTok}[1]{\textcolor[rgb]{0.25,0.63,0.44}{{#1}}}
    \newcommand{\BaseNTok}[1]{\textcolor[rgb]{0.25,0.63,0.44}{{#1}}}
    \newcommand{\FloatTok}[1]{\textcolor[rgb]{0.25,0.63,0.44}{{#1}}}
    \newcommand{\CharTok}[1]{\textcolor[rgb]{0.25,0.44,0.63}{{#1}}}
    \newcommand{\StringTok}[1]{\textcolor[rgb]{0.25,0.44,0.63}{{#1}}}
    \newcommand{\CommentTok}[1]{\textcolor[rgb]{0.38,0.63,0.69}{\textit{{#1}}}}
    \newcommand{\OtherTok}[1]{\textcolor[rgb]{0.00,0.44,0.13}{{#1}}}
    \newcommand{\AlertTok}[1]{\textcolor[rgb]{1.00,0.00,0.00}{\textbf{{#1}}}}
    \newcommand{\FunctionTok}[1]{\textcolor[rgb]{0.02,0.16,0.49}{{#1}}}
    \newcommand{\RegionMarkerTok}[1]{{#1}}
    \newcommand{\ErrorTok}[1]{\textcolor[rgb]{1.00,0.00,0.00}{\textbf{{#1}}}}
    \newcommand{\NormalTok}[1]{{#1}}
    
    % Additional commands for more recent versions of Pandoc
    \newcommand{\ConstantTok}[1]{\textcolor[rgb]{0.53,0.00,0.00}{{#1}}}
    \newcommand{\SpecialCharTok}[1]{\textcolor[rgb]{0.25,0.44,0.63}{{#1}}}
    \newcommand{\VerbatimStringTok}[1]{\textcolor[rgb]{0.25,0.44,0.63}{{#1}}}
    \newcommand{\SpecialStringTok}[1]{\textcolor[rgb]{0.73,0.40,0.53}{{#1}}}
    \newcommand{\ImportTok}[1]{{#1}}
    \newcommand{\DocumentationTok}[1]{\textcolor[rgb]{0.73,0.13,0.13}{\textit{{#1}}}}
    \newcommand{\AnnotationTok}[1]{\textcolor[rgb]{0.38,0.63,0.69}{\textbf{\textit{{#1}}}}}
    \newcommand{\CommentVarTok}[1]{\textcolor[rgb]{0.38,0.63,0.69}{\textbf{\textit{{#1}}}}}
    \newcommand{\VariableTok}[1]{\textcolor[rgb]{0.10,0.09,0.49}{{#1}}}
    \newcommand{\ControlFlowTok}[1]{\textcolor[rgb]{0.00,0.44,0.13}{\textbf{{#1}}}}
    \newcommand{\OperatorTok}[1]{\textcolor[rgb]{0.40,0.40,0.40}{{#1}}}
    \newcommand{\BuiltInTok}[1]{{#1}}
    \newcommand{\ExtensionTok}[1]{{#1}}
    \newcommand{\PreprocessorTok}[1]{\textcolor[rgb]{0.74,0.48,0.00}{{#1}}}
    \newcommand{\AttributeTok}[1]{\textcolor[rgb]{0.49,0.56,0.16}{{#1}}}
    \newcommand{\InformationTok}[1]{\textcolor[rgb]{0.38,0.63,0.69}{\textbf{\textit{{#1}}}}}
    \newcommand{\WarningTok}[1]{\textcolor[rgb]{0.38,0.63,0.69}{\textbf{\textit{{#1}}}}}
    
    
    % Define a nice break command that doesn't care if a line doesn't already
    % exist.
    \def\br{\hspace*{\fill} \\* }
    % Math Jax compatability definitions
    \def\gt{>}
    \def\lt{<}
    % Document parameters
    \title{DJSpresentation}
    
    
    

    % Pygments definitions
    
\makeatletter
\def\PY@reset{\let\PY@it=\relax \let\PY@bf=\relax%
    \let\PY@ul=\relax \let\PY@tc=\relax%
    \let\PY@bc=\relax \let\PY@ff=\relax}
\def\PY@tok#1{\csname PY@tok@#1\endcsname}
\def\PY@toks#1+{\ifx\relax#1\empty\else%
    \PY@tok{#1}\expandafter\PY@toks\fi}
\def\PY@do#1{\PY@bc{\PY@tc{\PY@ul{%
    \PY@it{\PY@bf{\PY@ff{#1}}}}}}}
\def\PY#1#2{\PY@reset\PY@toks#1+\relax+\PY@do{#2}}

\expandafter\def\csname PY@tok@w\endcsname{\def\PY@tc##1{\textcolor[rgb]{0.73,0.73,0.73}{##1}}}
\expandafter\def\csname PY@tok@c\endcsname{\let\PY@it=\textit\def\PY@tc##1{\textcolor[rgb]{0.25,0.50,0.50}{##1}}}
\expandafter\def\csname PY@tok@cp\endcsname{\def\PY@tc##1{\textcolor[rgb]{0.74,0.48,0.00}{##1}}}
\expandafter\def\csname PY@tok@k\endcsname{\let\PY@bf=\textbf\def\PY@tc##1{\textcolor[rgb]{0.00,0.50,0.00}{##1}}}
\expandafter\def\csname PY@tok@kp\endcsname{\def\PY@tc##1{\textcolor[rgb]{0.00,0.50,0.00}{##1}}}
\expandafter\def\csname PY@tok@kt\endcsname{\def\PY@tc##1{\textcolor[rgb]{0.69,0.00,0.25}{##1}}}
\expandafter\def\csname PY@tok@o\endcsname{\def\PY@tc##1{\textcolor[rgb]{0.40,0.40,0.40}{##1}}}
\expandafter\def\csname PY@tok@ow\endcsname{\let\PY@bf=\textbf\def\PY@tc##1{\textcolor[rgb]{0.67,0.13,1.00}{##1}}}
\expandafter\def\csname PY@tok@nb\endcsname{\def\PY@tc##1{\textcolor[rgb]{0.00,0.50,0.00}{##1}}}
\expandafter\def\csname PY@tok@nf\endcsname{\def\PY@tc##1{\textcolor[rgb]{0.00,0.00,1.00}{##1}}}
\expandafter\def\csname PY@tok@nc\endcsname{\let\PY@bf=\textbf\def\PY@tc##1{\textcolor[rgb]{0.00,0.00,1.00}{##1}}}
\expandafter\def\csname PY@tok@nn\endcsname{\let\PY@bf=\textbf\def\PY@tc##1{\textcolor[rgb]{0.00,0.00,1.00}{##1}}}
\expandafter\def\csname PY@tok@ne\endcsname{\let\PY@bf=\textbf\def\PY@tc##1{\textcolor[rgb]{0.82,0.25,0.23}{##1}}}
\expandafter\def\csname PY@tok@nv\endcsname{\def\PY@tc##1{\textcolor[rgb]{0.10,0.09,0.49}{##1}}}
\expandafter\def\csname PY@tok@no\endcsname{\def\PY@tc##1{\textcolor[rgb]{0.53,0.00,0.00}{##1}}}
\expandafter\def\csname PY@tok@nl\endcsname{\def\PY@tc##1{\textcolor[rgb]{0.63,0.63,0.00}{##1}}}
\expandafter\def\csname PY@tok@ni\endcsname{\let\PY@bf=\textbf\def\PY@tc##1{\textcolor[rgb]{0.60,0.60,0.60}{##1}}}
\expandafter\def\csname PY@tok@na\endcsname{\def\PY@tc##1{\textcolor[rgb]{0.49,0.56,0.16}{##1}}}
\expandafter\def\csname PY@tok@nt\endcsname{\let\PY@bf=\textbf\def\PY@tc##1{\textcolor[rgb]{0.00,0.50,0.00}{##1}}}
\expandafter\def\csname PY@tok@nd\endcsname{\def\PY@tc##1{\textcolor[rgb]{0.67,0.13,1.00}{##1}}}
\expandafter\def\csname PY@tok@s\endcsname{\def\PY@tc##1{\textcolor[rgb]{0.73,0.13,0.13}{##1}}}
\expandafter\def\csname PY@tok@sd\endcsname{\let\PY@it=\textit\def\PY@tc##1{\textcolor[rgb]{0.73,0.13,0.13}{##1}}}
\expandafter\def\csname PY@tok@si\endcsname{\let\PY@bf=\textbf\def\PY@tc##1{\textcolor[rgb]{0.73,0.40,0.53}{##1}}}
\expandafter\def\csname PY@tok@se\endcsname{\let\PY@bf=\textbf\def\PY@tc##1{\textcolor[rgb]{0.73,0.40,0.13}{##1}}}
\expandafter\def\csname PY@tok@sr\endcsname{\def\PY@tc##1{\textcolor[rgb]{0.73,0.40,0.53}{##1}}}
\expandafter\def\csname PY@tok@ss\endcsname{\def\PY@tc##1{\textcolor[rgb]{0.10,0.09,0.49}{##1}}}
\expandafter\def\csname PY@tok@sx\endcsname{\def\PY@tc##1{\textcolor[rgb]{0.00,0.50,0.00}{##1}}}
\expandafter\def\csname PY@tok@m\endcsname{\def\PY@tc##1{\textcolor[rgb]{0.40,0.40,0.40}{##1}}}
\expandafter\def\csname PY@tok@gh\endcsname{\let\PY@bf=\textbf\def\PY@tc##1{\textcolor[rgb]{0.00,0.00,0.50}{##1}}}
\expandafter\def\csname PY@tok@gu\endcsname{\let\PY@bf=\textbf\def\PY@tc##1{\textcolor[rgb]{0.50,0.00,0.50}{##1}}}
\expandafter\def\csname PY@tok@gd\endcsname{\def\PY@tc##1{\textcolor[rgb]{0.63,0.00,0.00}{##1}}}
\expandafter\def\csname PY@tok@gi\endcsname{\def\PY@tc##1{\textcolor[rgb]{0.00,0.63,0.00}{##1}}}
\expandafter\def\csname PY@tok@gr\endcsname{\def\PY@tc##1{\textcolor[rgb]{1.00,0.00,0.00}{##1}}}
\expandafter\def\csname PY@tok@ge\endcsname{\let\PY@it=\textit}
\expandafter\def\csname PY@tok@gs\endcsname{\let\PY@bf=\textbf}
\expandafter\def\csname PY@tok@gp\endcsname{\let\PY@bf=\textbf\def\PY@tc##1{\textcolor[rgb]{0.00,0.00,0.50}{##1}}}
\expandafter\def\csname PY@tok@go\endcsname{\def\PY@tc##1{\textcolor[rgb]{0.53,0.53,0.53}{##1}}}
\expandafter\def\csname PY@tok@gt\endcsname{\def\PY@tc##1{\textcolor[rgb]{0.00,0.27,0.87}{##1}}}
\expandafter\def\csname PY@tok@err\endcsname{\def\PY@bc##1{\setlength{\fboxsep}{0pt}\fcolorbox[rgb]{1.00,0.00,0.00}{1,1,1}{\strut ##1}}}
\expandafter\def\csname PY@tok@kc\endcsname{\let\PY@bf=\textbf\def\PY@tc##1{\textcolor[rgb]{0.00,0.50,0.00}{##1}}}
\expandafter\def\csname PY@tok@kd\endcsname{\let\PY@bf=\textbf\def\PY@tc##1{\textcolor[rgb]{0.00,0.50,0.00}{##1}}}
\expandafter\def\csname PY@tok@kn\endcsname{\let\PY@bf=\textbf\def\PY@tc##1{\textcolor[rgb]{0.00,0.50,0.00}{##1}}}
\expandafter\def\csname PY@tok@kr\endcsname{\let\PY@bf=\textbf\def\PY@tc##1{\textcolor[rgb]{0.00,0.50,0.00}{##1}}}
\expandafter\def\csname PY@tok@bp\endcsname{\def\PY@tc##1{\textcolor[rgb]{0.00,0.50,0.00}{##1}}}
\expandafter\def\csname PY@tok@fm\endcsname{\def\PY@tc##1{\textcolor[rgb]{0.00,0.00,1.00}{##1}}}
\expandafter\def\csname PY@tok@vc\endcsname{\def\PY@tc##1{\textcolor[rgb]{0.10,0.09,0.49}{##1}}}
\expandafter\def\csname PY@tok@vg\endcsname{\def\PY@tc##1{\textcolor[rgb]{0.10,0.09,0.49}{##1}}}
\expandafter\def\csname PY@tok@vi\endcsname{\def\PY@tc##1{\textcolor[rgb]{0.10,0.09,0.49}{##1}}}
\expandafter\def\csname PY@tok@vm\endcsname{\def\PY@tc##1{\textcolor[rgb]{0.10,0.09,0.49}{##1}}}
\expandafter\def\csname PY@tok@sa\endcsname{\def\PY@tc##1{\textcolor[rgb]{0.73,0.13,0.13}{##1}}}
\expandafter\def\csname PY@tok@sb\endcsname{\def\PY@tc##1{\textcolor[rgb]{0.73,0.13,0.13}{##1}}}
\expandafter\def\csname PY@tok@sc\endcsname{\def\PY@tc##1{\textcolor[rgb]{0.73,0.13,0.13}{##1}}}
\expandafter\def\csname PY@tok@dl\endcsname{\def\PY@tc##1{\textcolor[rgb]{0.73,0.13,0.13}{##1}}}
\expandafter\def\csname PY@tok@s2\endcsname{\def\PY@tc##1{\textcolor[rgb]{0.73,0.13,0.13}{##1}}}
\expandafter\def\csname PY@tok@sh\endcsname{\def\PY@tc##1{\textcolor[rgb]{0.73,0.13,0.13}{##1}}}
\expandafter\def\csname PY@tok@s1\endcsname{\def\PY@tc##1{\textcolor[rgb]{0.73,0.13,0.13}{##1}}}
\expandafter\def\csname PY@tok@mb\endcsname{\def\PY@tc##1{\textcolor[rgb]{0.40,0.40,0.40}{##1}}}
\expandafter\def\csname PY@tok@mf\endcsname{\def\PY@tc##1{\textcolor[rgb]{0.40,0.40,0.40}{##1}}}
\expandafter\def\csname PY@tok@mh\endcsname{\def\PY@tc##1{\textcolor[rgb]{0.40,0.40,0.40}{##1}}}
\expandafter\def\csname PY@tok@mi\endcsname{\def\PY@tc##1{\textcolor[rgb]{0.40,0.40,0.40}{##1}}}
\expandafter\def\csname PY@tok@il\endcsname{\def\PY@tc##1{\textcolor[rgb]{0.40,0.40,0.40}{##1}}}
\expandafter\def\csname PY@tok@mo\endcsname{\def\PY@tc##1{\textcolor[rgb]{0.40,0.40,0.40}{##1}}}
\expandafter\def\csname PY@tok@ch\endcsname{\let\PY@it=\textit\def\PY@tc##1{\textcolor[rgb]{0.25,0.50,0.50}{##1}}}
\expandafter\def\csname PY@tok@cm\endcsname{\let\PY@it=\textit\def\PY@tc##1{\textcolor[rgb]{0.25,0.50,0.50}{##1}}}
\expandafter\def\csname PY@tok@cpf\endcsname{\let\PY@it=\textit\def\PY@tc##1{\textcolor[rgb]{0.25,0.50,0.50}{##1}}}
\expandafter\def\csname PY@tok@c1\endcsname{\let\PY@it=\textit\def\PY@tc##1{\textcolor[rgb]{0.25,0.50,0.50}{##1}}}
\expandafter\def\csname PY@tok@cs\endcsname{\let\PY@it=\textit\def\PY@tc##1{\textcolor[rgb]{0.25,0.50,0.50}{##1}}}

\def\PYZbs{\char`\\}
\def\PYZus{\char`\_}
\def\PYZob{\char`\{}
\def\PYZcb{\char`\}}
\def\PYZca{\char`\^}
\def\PYZam{\char`\&}
\def\PYZlt{\char`\<}
\def\PYZgt{\char`\>}
\def\PYZsh{\char`\#}
\def\PYZpc{\char`\%}
\def\PYZdl{\char`\$}
\def\PYZhy{\char`\-}
\def\PYZsq{\char`\'}
\def\PYZdq{\char`\"}
\def\PYZti{\char`\~}
% for compatibility with earlier versions
\def\PYZat{@}
\def\PYZlb{[}
\def\PYZrb{]}
\makeatother


    % Exact colors from NB
    \definecolor{incolor}{rgb}{0.0, 0.0, 0.5}
    \definecolor{outcolor}{rgb}{0.545, 0.0, 0.0}



    
    % Prevent overflowing lines due to hard-to-break entities
    \sloppy 
    % Setup hyperref package
    \hypersetup{
      breaklinks=true,  % so long urls are correctly broken across lines
      colorlinks=true,
      urlcolor=urlcolor,
      linkcolor=linkcolor,
      citecolor=citecolor,
      }
    % Slightly bigger margins than the latex defaults
    
    \geometry{verbose,tmargin=1in,bmargin=1in,lmargin=1in,rmargin=1in}
    
    

    \begin{document}
    
    
    \maketitle
    
    

    
    \hypertarget{dynamic-joint-stiffness-of-the-ankle-in-human-gait-reviewing-and-insights-of-the-data-from-literature-and-clinical-databases}{%
\section{Dynamic Joint Stiffness of the ankle in human gait: Reviewing
and insights of the Data from literature and clinical
Databases}\label{dynamic-joint-stiffness-of-the-ankle-in-human-gait-reviewing-and-insights-of-the-data-from-literature-and-clinical-databases}}

    \hypertarget{table-of-contents}{%
\section{Table of Contents}\label{table-of-contents}}

\begin{itemize}
\tightlist
\item
  Section \ref{dynamic-joint-stiffness-of-the-ankle-in-human-gait-reviewing-and-insights-of-the-data-from-literature-and-clinical-databases}

  \begin{itemize}
  \tightlist
  \item
    Section \ref{highlights}
  \item
    Section \ref{introduction}
  \item
    Section \ref{methods}
  \item
    Section \ref{results}
  \item
    Section \ref{discussion}
  \item
    Section \ref{future-work}
  \end{itemize}
\end{itemize}

    \hypertarget{highlights}{%
\subsection{Highlights}\label{highlights}}

    \begin{itemize}
\tightlist
\item
  A methodological search was made to find and collect datasets suitable
  to perform a dynamical analysis in the ankle joint.
\item
  A biomechanical analysis was performed for the ankle joint from
  childhood through adulthood in order to see differences in their
  dynamic patterns.
\item
  An enhanced method for detecting instances in ankle DJS was proposed.
\item
  Different gait genders were considered in the study.
\item
  The information collected could be useful for prosthesis designers and
  control engineers.
\end{itemize}

    \hypertarget{state-of-the-art.}{%
\subsection{State of the art.}\label{state-of-the-art.}}

    \hypertarget{introduction}{%
\subsubsection{Introduction}\label{introduction}}

\begin{itemize}
\tightlist
\item
  Concept adopted by \cite{Davis1996} in which they defined this concept
  as: The resistance that the muscle and other soft tissue structures
  that cross a joint offer during gait in response to an applied moment.
\end{itemize}

    \begin{itemize}
\tightlist
\item
  The importance of the DJS for prosthetic application in the ankle
  joint was coined by \cite{Hansen2004}. There, important insights were
  revealed. Another studies such as \cite{Safaeepour2014} mention the
  same aspects. However, a proposed concept of quasi-stiffness in active
  prostheses was given by \cite{Rouse2013a}.
\end{itemize}

\begin{verbatim}
<div class="column">
    <img src="./figures/fastQuasi.png" alt="Fast" style="width:40%">
</div>
<div class="column">
    <img src="./figures/NormalQuasi.png" alt="Normal" style="width:40%">
</div>
<div class="column">
    <img src="./figures/SlowQuasi.png" alt="Slow" style="width:40%">
</div>
\end{verbatim}

    Later, \cite{Crenna2011} established a method for linearized the
instances of the ankle DJS.

Where the Threshold 1 is set when the moment increased up to 5\%, the
Threshold 2 is set at 95\% of the maximum moment, for the maximum limit
of the Early Response Phase and the beginning of the Large Response
Phase, they apply the following equation:

\begin{equation}
S_{i} = (M_{i+1}-M{i})/(\theta_{i+1}-\theta{i}) >= 1.7
\end{equation}

Now, applying the limits to the function we can obtain the coefficients
with the form: \(Y=a+bx\) on each method.

Also, they made the energetic analysis of this human joint, as the image
shows.

To calculate the work along the x-axis, the Simpon's Rule was applied.
Mathematically this rule is represented as: \begin{equation}
Area=\int_{a}^{b}f(x)dx\thickapprox\frac{\triangle x}{3}[y_{0}+4\sum odds+2\sum even+y_{n}]
\end{equation}

\hypertarget{types-of-quasi-stiffness}{%
\subsubsection{Types of
Quasi-stiffness}\label{types-of-quasi-stiffness}}

    \begin{itemize}
\tightlist
\item
  \cite{Shamaei2013} proposed the identification of the key independent
  parameters needed to predict ankle quasi-stiffness and propulsive work
  and also the functional form of each correlation.
\end{itemize}

    \hypertarget{motivation}{%
\subsection{Motivation}\label{motivation}}

\begin{itemize}
\item
  Dynamic Joint Stiffness can be used either for studying pathologic
  abnormalities and to develop prostheses and orthoses
  \cite{Sanchis-Sales2016}.
\item
  Many studies support the DJS analysis as the primary input for any
  prosthetic or orthetic device (e.g.
  \cite{Au2009, Herr2012, Dong2017, Holgate2017}).
\end{itemize}

\begin{itemize}
\item
  On the other hand, the DJS is use to contrast any pathological with
  non-pathological behavior (e.g. \cite{Aleixo2015, Wang2012}) and
  predict irregular patterns in the ankle dynamics.
\item
  The DJS of the human ankle has been studied to find dynamical patterns
  such that changes in gender \cite{Gabriel2008}, gender and age
  \cite{Crenna2011} , and anthropometrics \cite{Powell2014}.
\item
  The importance of reusing the data is highly remarkable by
  \cite{white2013} and for that reason would be helpful for the research
  community to show proof of homogeneity in the data.
\item
  Many databases were created with the aim of collecting human gait data
  as \cite{Tirosh2010, Mandery2016, Hu2018}, but they are not useful at
  all for dynamic analysis purposes.
\item
  Mention the importance for AI.
\end{itemize}

    \hypertarget{methods}{%
\subsection{Methods}\label{methods}}

    \hypertarget{literature-search}{%
\subsubsection{Literature Search}\label{literature-search}}

    \hypertarget{online-gait-database-search}{%
\subsubsection{Online Gait Database
search}\label{online-gait-database-search}}

Criteria of the data to be processed in the study of the ankle DJS{}

ID

Requirement

1

It should be related with regular gait

2

Not older than 10 years ago.

3

Definitely, it must contain kinetic information.

4

The format given should be familiar with standard data in human gait,
e.g., C3D, CSV, txt.

5

It should be well organized and unrestrictable

6

It must contain the methods of how it was measured.

    \hypertarget{selection-of-the-data}{%
\subsubsection{Selection of the data}\label{selection-of-the-data}}

Clinical Databases found in the web.

1

2

3

4

5

6

1

Physiobank databases {}

\textless{}100

3D motion

.dat

X

-

-

O

O

O

0

2

Clinical Gait analysis {}

13

3D motion, EMG

.m .txt .xls

O

X

O

O

O

O

1

3

Koroibot {}

544

3D motion, pressure, EMG

mmm, c3d, csv

O

O

-

O

O

O

10

4

Motion Capture HDM05 {}

\textless{} 50

3D Motion

c3d, amc

O

O

X

O

O

O

0

5

CMU Graphics Lab Motion Capture {}

\textless{} 50

3D motion

tvd, c3d, amc, mpg

O

-

X

O

O

-

0

6

MAREA Dataset {}

2

Acelerometers

.m

O

O

X

-

O

O

1

7

ENABL3S {}

1

EMG, GONIO, IMU

csv

O

O

X

O

O

O

1

8

Gaitabase {}

1

N.A.

N.A.

X

X

X

X

X

X

1

Note: The table represents the level of accomplishment in Table
{[}table:Criteria{]}, where (O) meets, (-) fulfill partially and (X)
does not meet the requirement.
\protect\hypertarget{table:Databases}{}{{[}table:Databases{]}}

    \hypertarget{methodology}{%
\subsection{Methodology}\label{methodology}}

We implemented the python programming language because it integrates
libraries that concerns with the PhD thesis, such as: Opensim for
biomechanical modeling, numpy for numerical computing, SciPy for
optimization techniques, bokeh for graphical representations, Scikit
learn to get regression models, Pandas to manage DataSets and jupyter to
present information in notebooks. In not all cases all the libraries
were implemented.

In that order of ideas, we describe the activities done with every study
found in the literature as well as in gait Databases, which are
mentioned below:

\begin{itemize}
\item Selecting, filtering and cleaning the data.
\item Obtaining physical variables and visualizing behaviors at the ankle
joint.
\begin{itemize}
\item Obtaining the integral as the area under a curve through
Simpson's Rule of ankle power and ankle DJS.
\item Visualizing the ankle powers and ankle DJS at different speeds, modes of walking, ages, etc.
\item Obtaining the work done (work absorbed, work produced and net work) through the integration of the power loop.
\end{itemize}
\item Configuring Dynamic Joint Stiffness variables, such as sub-phases of gait and linear slopes. 
\begin{itemize}
\item Plotting the angle-moment loop at different speeds, modes and ages.
\item Determining the dorsi-flexion limit point, the bottom and the top
threshold through two ways:
\begin{itemize}
\item As mentioned by Crenna and Frigo \cite{Crenna2011}.
\item As mentioned by Shamaei et al. \cite{Shamaei2013}.
\end{itemize} 
\end{itemize}
\item Making the linear regression at each sub-phase, doing these at different ways, such as:
\begin{itemize}
\item Using Least Squared Methods with Scikit-learn tool.
\item Using Non-linear Least Squared Methods through SciPy Library.
\end{itemize}
\end{itemize}

    \hypertarget{results}{%
\subsection{Results}\label{results}}

    Year

Title

Authors

Subjects

Method

General Observations

Type of file

2006

Regression analysis of gait parameters with speed in normal children
walking at self-selected speeds.

\cite{Stansfield2006}

16

Ground reaction forces (Kistler Instruments, AG Winterthur, Switzerland)
and motion analysis data (five camera stystem---Vicon, Oxford Metrics
Group, Oxford, UK) were collected at 50 Hz. Inverse dynamics were
calculated with Vicon Clinical Manager.

Children between the ages of 7 and 12 years (eight boys and eight girls)
each year for 5 consecutive years. Children walked barefoot at
self-selected normal velocities. GRF were calculated for approximately
three sets of data for each leg for each child for each of the 5
consecutive years of the study (a total of 457 trials). Weight, height
and leg length (greater trochanter to lateral malleolus) data were
recorded.

.xls

2008

The effect of walking speed on the gait of typically developing children

\cite{Schwartz2008}

83

12-camera Vicon MX system (Vicon,Oxford, UK) operating at 120 Hz. Ground
reaction forces were recorded using four force plates (AMTI, Watertown,
MA), sampled at 1080 Hz. Surface EMG signals were also measured.

Three-dimensional gait data was collected on 83 subjects who were given
general instructions to walk at very slow, slow, self-selected
comfortable (free), and fast walking speeds during a single testing
session.

.xls

2011

A multiple-task gait analysis approach: Kinematic, kinetic and EMG
reference data for healthy young and adult subjects

\cite{Bovi2011a}

40

3D kinematics was measured using a 9-cameras SMART-E motion capture
system (BTS). Two force plates (Kistler, Winterthur, Switzerland), at
960 Hz sampling frequency, provided ground reaction forces (GRFs)

20 subjects included in the adult group (aged from 22 to 72 years, mean
43.1 15.4; body mass 68.5 15.8 kg; height 1.71 0.10 m; 9 males, 11
females) and 20 in the young group (aged from 6 to 17 years, mean 10.8
3.2; body mass 41.4 15.5 kg; height 1.47 0.20 m; 9 males, 11 females).

.xls

2015

An elaborate data set on human gait and the effect of mechanical
perturbations.

\cite{Moore2015}

15

A R-Mill treadmill which has dual 6 degree of freedom force plates,
independent belts for each foot, along with lateral translation and
pitch rotation capabilities. A 10 Osprey camera motion capture system.
Four ADXL330 Triple Axis Accelerometer Breakout boards attached to the
treadmill.

Four females and eleven males with an average age of 244 years, height
of 1.750.09 m, mass of 7413 kg participated in the study.

.txt

2018

Benchmark Datasets for Bilateral Lower-Limb Neuromechanical Signals from
Wearable Sensors during Unassisted Locomotion in Able-Bodied Individuals

\cite{Hu2018}

10

Subjects were instrumented with wearable sensors to measure bilateral
lower limb muscle activity and joint and limb kinematics. EMG signals
were recorded using bipolar surface electrodes (DE2.1; Delsys, Boston,
MA, USA)

10 subjects (seven male, three female); 25.5 2 years; 174 12 cm; 70 14
kg without any gait impairments were recruited.

csv

2017

Evaluation of the performance of accelerometer-based gait event
detection algorithms in different real-world scenarios using the MAREA
gait database

\cite{Khandelwal2017}

20

Each subject had a 3-axes Shimmer3 (Shimmer Research, Dublin, Ireland)
accelerometer (8 g) attached to their waist, left wrist and left and
right ankles using elastic bands and velcro straps.

20 healthy adults:12 males and 8 females, average age: 33.4 7 years,
average mass: 73.2 10.9 kg, average height: 172.6 9.5 cm.

.mat

    \hypertarget{data}{%
\subsection{Data}\label{data}}

\href{Ferrarin\%20Data-Presentation.ipynb}{A multiple-task gait analysis
approach: Kinematic, kinetic and EMG reference data for healthy young
and adult subjects}

\href{Stansfield-and-Schwaltz.ipynb}{Ankle DJS analyze in children at
different velocities}

    \hypertarget{discussion}{%
\subsection{Discussion}\label{discussion}}

\begin{itemize}
\item
  Recent online databases for human gait need to be fed and could be
  focused on to human improvements, due to the search demonstrated that
  most of databases are oriented to other purposes as: motion, robotics,
  etc.
\item
  Recent techniques showed to set the instances in the quasi-stiffness
  slope are inefficient for a wide range of speed, It is recommended to
  apply the curvature in those cases.
\item
  Comparing data between similar subjects shows regularities in terms of
  the type of quasi-stiffness given.
\item
  The DJS in children at lower speeds and irregular walking (e.g.~toes,
  heel, and descending) showed a clockwise behavior.
\item
  A narrow quasi-stiffness was seen in children at lower speed. On the
  other hand, at higher speed in children a rounded DJS was observed in
  them.
\end{itemize}

    \hypertarget{future-work}{%
\subsection{Future Work}\label{future-work}}

\begin{itemize}
\tightlist
\item
  Conssidering the statistics in the overall data.
\item
  A possible classifier for a pathological DJS can be done if more data
  is collected.
\item
  A meta analysis for all types of data might be useful.
\item
  Finishing the data analysis for the rest of the data.
\end{itemize}

    \hypertarget{references}{%
\section{References}\label{references}}

(Davis and DeLuca, 1996) Davis Roy B. and DeLuca Peter A., ``\emph{Gait
characterization via dynamic joint stiffness}'', Gait and Posture,
vol.~4, number 3, pp.~224--231, 1996.

(Hansen, Childress et al., 2004) Hansen Andrew H., Childress Dudley S.,
Miff Steve C. et al., ``\emph{The human ankle during walking:
Implications for design of biomimetic ankle prostheses}'', Journal of
Biomechanics, vol.~37, number , pp.~1467--1474, 2004.

(Safaeepour, Esteki et al., 2014) Safaeepour Zahra, Esteki Ali, Ghomshe
Farhad T. et al., ``\emph{Quantitative analysis of human ankle
characteristics at different gait phases and speeds for utilizing in
ankle-foot prosthetic design}'', BioMedical Engineering Online, vol.~13,
number 1, pp.~1--8, 2014.

(Rouse, Gregg et al., 2013) Rouse Elliott J., Gregg Robert D., Hargrove
Levi J. et al., ``\emph{The difference between stiffness and
quasi-stiffness in the context of biomechanical modeling}'', IEEE
Transactions on Biomedical Engineering, vol.~60, number 2, pp.~562--568,
2013.

(Crenna and Frigo, 2011) Crenna Paolo and Frigo Carlo, ``\emph{Dynamics
of the ankle joint analyzed through moment-angle loops during human
walking: gender and age effects.}'', Human movement science, vol.~30,
number 6, pp.~1185--98, dec 2011.
\href{http://www.sciencedirect.com/science/article/pii/S016794571100073X}{online}

(Shamaei, Sawicki et al., 2013) Shamaei Kamran, Sawicki Gregory S. and
Dollar Aaron M., ``\emph{Estimation of Quasi-Stiffness and Propulsive
Work of the Human Ankle in the Stance Phase of Walking}'', PLoS ONE,
vol.~8, number 3, pp. , 2013.

(Sanchis-Sales, Sancho-Bru et al., 2016) Sanchis-Sales Enrique,
Sancho-Bru Joaquin L, Roda-Sales Alba et al., ``\emph{Dynamic Flexion
Stiffness of Foot Joints During Walking.}'', Journal of the American
Podiatric Medical Association, vol.~106, number 1, pp.~37--46, 2016.
\href{http://www.ncbi.nlm.nih.gov/pubmed/26895359}{online}

(Au, Weber et al., 2009) Au S K, Weber Jeff and Herr Hugh,
``\emph{Powered Ankle--Foot Prosthesis Improves Walking Metabolic
Economy}'', IEEE Transactions on Robotics, vol.~25, number 1,
pp.~51--66, 2009.
\href{http://ieeexplore.ieee.org/document/4738392/}{online}

(Herr and Grabowski, 2012) Herr H M and Grabowski a. M, ``\emph{Bionic
ankle-foot prosthesis normalizes walking gait for persons with leg
amputation}'', Proceedings of the Royal Society B: Biological Sciences,
vol.~279, number 1728, pp.~457--464, 2012.

(Dong, Ge et al., 2017) Dong Dianbiao, Ge Wenjie, Liu Shumin et al.,
``\emph{Design and optimization of a powered ankle-foot prosthesis using
a geared five-bar spring mechanism}'', International Journal of Advanced
Robotic Systems, vol.~14, number 3, pp.~172988141770454, may 2017.
\href{http://journals.sagepub.com/doi/10.1177/1729881417704545}{online}

(Holgate, Sugar et al., 2017) R. Holgate, T. Sugar, A. Nash et al.,
``\emph{A Passive Ankle-Foot Prosthesis With Energy Return to Mimic
Able-Bodied Gait}'', Volume 5A: 41st Mechanisms and Robotics Conference,
aug 2017.
\href{http://proceedings.asmedigitalcollection.asme.org/proceeding.aspx?doi=10.1115/DETC2017-67192}{online}

(Aleixo, \{Vaz Patto\} et al., 2015) Aleixo P, \{Vaz Patto\} J, Roupa I
et al., ``\emph{Dynamic joint stiffness of the ankle in healthy and
rheumatoid arthritis postmenopausal women}'', Gait \& Posture, vol.~60,
number October 2017, pp.~225--234, 2015.
\href{https://doi.org/10.1016/j.gaitpost.2017.12.008}{online}

(Wang, Brostr\{"\{o\}\}m et al., 2012) Wang Ruoli, Brostr\{"\{o\}\}m Eva
W., Esbj\{"\{o\}\}rnsson Anna Clara et al., ``\emph{Analytical
decomposition can help to interpret ankle joint moment-angle
relationship}'', Journal of Electromyography and Kinesiology, vol.~22,
number 4, pp.~566--574, 2012.
\href{http://dx.doi.org/10.1016/j.jelekin.2012.04.005}{online}

(Gabriel, Abrantes et al., 2008) Gabriel Ronaldo C, Abrantes
Jo\{\textasciitilde{}\{a\}\}o, Granata Kevin et al., ``\emph{Dynamic
joint stiffness of the ankle during walking: gender-related
differences.}'', Physical therapy in sport : official journal of the
Association of Chartered Physiotherapists in Sports Medicine, vol.~9,
number 1, pp.~16--24, feb 2008.
\href{http://www.sciencedirect.com/science/article/pii/S1466853X07000776}{online}

(Powell, Williams et al., 2014) Powell Douglas W., Williams D. S Blaise,
Windsor Brett et al., ``\emph{Ankle work and dynamic joint stiffness in
high- compared to low-arched athletes during a barefoot running task}'',
Human Movement Science, vol.~34, number 1, pp.~147--156, 2014.
\href{http://dx.doi.org/10.1016/j.humov.2014.01.007}{online}

(White, Baldridge et al., 2013) White Ethan, Baldridge Elita, Brym
Zachary et al., ``\emph{Nine simple ways to make it easier to (re)use
your data}'', Ideas in Ecology and Evolution, vol.~6, number 2, pp. ,
2013.
\href{http://library.queensu.ca/ojs/index.php/IEE/article/view/4608}{online}

(Tirosh, Baker et al., 2010) Tirosh Oren, Baker Richard and McGinley
Jenny, ``\emph{GaitaBase: Web-based repository system for gait
analysis}'', Computers in Biology and Medicine, vol.~40, number 2,
pp.~201--207, 2010.
\href{http://dx.doi.org/10.1016/j.compbiomed.2009.11.016}{online}

(Mandery, Terlemez et al., 2016) Mandery Christian, Terlemez
\{"\{O\}\}mer, Do Martin et al., ``\emph{Unifying Representations and
Large-Scale Whole-Body Motion Databases for Studying Human Motion}'',
IEEE Transactions on Robotics, vol.~32, number 4, pp.~796--809, 2016.

(Hu, Rouse et al., 2018) Hu Blair, Rouse Elliott and Hargrove Levi,
``\emph{Benchmark Datasets for Bilateral Lower-Limb Neuromechanical
Signals from Wearable Sensors during Unassisted Locomotion in
Able-Bodied Individuals}'', Frontiers in Robotics and AI, vol.~5, number
February, pp.~1--5, 2018.
\href{http://journal.frontiersin.org/article/10.3389/frobt.2018.00014/full}{online}

(Stansfield, Hillman et al., 2006) Stansfield B. W., Hillman S. J.,
Hazlewood M. E. et al., ``\emph{Regression analysis of gait parameters
with speed in normal children walking at self-selected speeds}'', Gait
and Posture, vol.~23, number 3, pp.~288--294, 2006.

(Schwartz, Rozumalski et al., 2008) Schwartz Michael H., Rozumalski Adam
and Trost Joyce P., ``\emph{The effect of walking speed on the gait of
typically developing children}'', Gait and Posture, vol.~41, number 3,
pp.~351--357, 2008.

(Bovi, Rabuffetti et al., 2011) Bovi Gabriele, Rabuffetti Marco,
Mazzoleni Paolo et al., ``\emph{A multiple-task gait analysis approach:
Kinematic, kinetic and EMG reference data for healthy young and adult
subjects}'', Gait and Posture, vol.~33, number 1, pp.~6--13, 2011.
\href{http://dx.doi.org/10.1016/j.gaitpost.2010.08.009}{online}

(Moore, Hnat et al., 2015) Moore Jason K., Hnat Sandra K. and van den
Bogert Antonie J., ``\emph{An elaborate data set on human gait and the
effect of mechanical perturbations}'', PeerJ, vol.~3, number April,
pp.~e918, 2015. \href{https://peerj.com/articles/918}{online}

(Khandelwal and Wickstr\{"\{o\}\}m, 2017) Khandelwal Siddhartha and
Wickstr\{"\{o\}\}m Nicholas, ``\emph{Evaluation of the performance of
accelerometer-based gait event detection algorithms in different
real-world scenarios using the MAREA gait database}'', Gait and Posture,
vol.~51, number , pp.~84--90, 2017.


    % Add a bibliography block to the postdoc
    
    
\bibliographystyle{ieeetr}
\bibliography{/home/eprietop/Dropbox/library.bib}

    
    \end{document}
